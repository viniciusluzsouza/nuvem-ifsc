
% This LaTeX was auto-generated from MATLAB code.
% To make changes, update the MATLAB code and republish this document.

\documentclass{article}
\usepackage{graphicx}
\usepackage{color}

\sloppy
\definecolor{lightgray}{gray}{0.5}
\setlength{\parindent}{0pt}

\begin{document}

    
    
\subsection*{Contents}

\begin{itemize}
\setlength{\itemsep}{-1ex}
   \item Experimento 1.2
\end{itemize}
\begin{verbatim}
%  Exemplos e Experimentos baseados no livro:
% DINIZ, P. S. R., DA SILVA, E. A. B., e LIMA NETTO, S. Processamento Digital de Sinais: Projeto e An?lise de Sistemas. 2. ed. Porto Alegre: Bookman, 2014. 976 p. ISBN 978-8582601235.
\end{verbatim}


\subsection*{Experimento 1.2}

\begin{verbatim}
fs = 10; % frequencia (Hz) de amostragem dos sinais
Ts = 1/fs; fase = 0;
time = 0:Ts:(1-Ts);
f1 = 3; % frequencia (Hz) do sinal s_1
f2 = 7; % frequencia (Hz) do sinal s_2
s_1 = cos(2*pi*f1*time+fase);
s_2 = cos(2*pi*f2*time+fase);
fsa = 1000; % frequ?ncia auxiliar de amostragem usada apenas para representa??o dos sinais originais
Tsa = 1/fsa;
time_aux = 0:Tsa:(1-Tsa);
figure(1);
stem(time,s_1,'ob');
hold on;
plot(time_aux, cos(2*pi*f1*time_aux+fase),'--k');
stem(time,s_2,'+r');
plot(time_aux, cos(2*pi*f2*time_aux+fase),'--m');
hold off;
legend('s_1 discreto','s_1 cont?nuo','s_2 discreto','s_2 cont?nuo')
\end{verbatim}

\includegraphics [width=4in]{experimento_1_2_01.eps}



\end{document}
    
